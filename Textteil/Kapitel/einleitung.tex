\chapter{Einleitung}
\label{ch:einleitung}

\section{Umfeld}
Diese \LaTeX -Vorlage ist zur Nutzung bei der Erstellung von Hausarbeiten und der Bachelor-/Master-Thesis vorgesehen. Dabei wurde sich auf den Leitfaden zur Erstellung wissenschaftlicher Arbeiten aus dem Februar 2018 bezogen.\autocite{leitfaden}

\section{Motivation}
Gerade in den letzten Jahren wurde der Leitfaden zur Erstellung wissenschaftlicher Arbeiten an der FOM (\nomenclature{FOM}{Hochschule für Oekonomie und Management}Hochschule für Oekonomie und Management) immer komplexer. Viele Vorschriften lassen sich nicht wie früher noch mit den \LaTeX\-Standardeinstellungen oder kleineren Anpassungen erreichen, sondern müssen, wie beispielsweise das Zitieren (siehe Kapitel \ref{ch:zitieren}, S. \pageref{ch:zitieren}).

Um das Anfertigen von Hausarbeiten etc. zu vereinfachen, wurde dieses Template angefangen. Es soll den Kommilitonen als Anhaltspunkt für neue Arbeiten dienen und durch gemeinsame Weiterarbeit zunehmend wachsen. Aufgrunddessen wird selbstverständlich in keinster Weise die Richtigkeit der Formatierung garantiert. Ich habe mir zwar größte Mühe gegeben, aber Fehler passieren immer. Bei gefundenen Fehlern würde ich mich sehr über Rückmeldung freuen. Der Fehler wird dann schnellstmöglich behoben und schon haben wir alle etwas davon. :-)

Das folgende Dokument erklärt einige Funktionen des Templates, und wie ihr diese mit möglichst wenig Aufwand nutzen könnt. Auch hier stehe ich (oder ggf. andere Nutzer des Templates) bestimmt für Fragen zur Verfügung.

\section{Abgrenzung}
In der Anleitung werde ich lediglich auf die Nutzung auf Windows-Geräten eingehen. Zwar wird sich die Syntax innerhalb der .tex Dateien nicht ändern, doch kann es sein, dass die erstmalige Installation sich unterscheidet.