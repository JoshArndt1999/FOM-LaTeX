\chapter{Zitieren}
\label{ch:zitieren}
Das Zitieren ist ein elementarer Bestandteil einer jeden wissenschaftlichen Arbeit. Um alle Formvorschriften einzuhalten empfehle ich, lediglich drei verschiedene Befehle zu nutzen:
\begin{description}
	\item[{\textbackslash cite[a][b]\{c\}}] gibt den Fußnoteninhalt nicht in einer Fußnote, sondern direkt im Text aus. Dies wird vor allem bei der Quellenangabe von Abbildungen verlangt.
	\item[{\textbackslash autocite[a][b]\{c\}}] gibt eine gewöhnliche Fußnote mit Quelle aus.
	\item[{\textbackslash directAutocite[a][b]\{c\}\{d\}}] erzeugt ein direktes Zitat mit zugehöriger Fußnote. Die Anführungszeichen werden je nach vorheriger Festlegung gewählt (hier standardmäßig Guillements).
\end{description}
Für die Parameter a, b, c und d gilt:
\begin{description}
	\item[a] ist das Präfix der Fußnote (z. B. \enquote{Vgl.}). Dieser Parameter ist optional.
	\item[b] ist die Seitenzahl, von der die Information(/Zitat) entnommen wurde. Auch dieser Parameter ist optional. Sollte er jedoch weggelassen werden, sollten die Klammern trotzdem geschrieben werden, da ansonsten ein Suffix nicht vom Präfix unterschieden werden kann. In diesem Fall werden die Klammern also lediglich freigelassen.
	\item[c] ist die ID der Quelle.
	\item[d] ist der Text, welcher innerhalb der Anführungszeichen aufgeführt werden soll.
\end{description}

Hier ein Beispiel:
\begin{lstlisting}
\directAutocite[12]{bsp}{Das hier ist ein Zitat}
\end{lstlisting}
Ausgabe: \directAutocite[12]{bsp}{Das hier ist ein Zitat}