% !TeX root = ../../main.tex

\chapter{Quelltext}

Wie ihr in der Anleitung nun schon einige Male gesehen habt, lässt sich auch Quelltext in die Arbeit integrieren. Hierzu schreibt man den einzufügenden Quelltext zwischen die Befehle \textbackslash begin\{listing\} und \textbackslash end\{listing\}

Der Quelltext wird nun eingefügt und mit Zeilennummern versehen. Lasst euch nicht irritieren, wenn der Code vom Editor als Befehl registriert wird (insbesondere wenn es sich um einzufügenden \LaTeX -Code handelt. Im resultierenden PDF ist es trotzdem als eingefügter Quelltext vorhanden. Es ist also lediglich die Anzeige im Editor fehlerhaft.
