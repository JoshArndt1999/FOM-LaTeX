\chapter{Installation}
\label{ch:installation}
Zur Nutzung von \LaTeX\ müssen zunächst einmal einige Dinge auf dem Rechner installiert werden. Es folgt eine kleine Anleitung, mit der auch das kein Problem sein sollte.

\section{MiKTeX}
MiKTeX ist neben TeX Live und MacTex eine der größten Distributionen, die es von TeX gibt. Für Windows Systeme ist diese sinnvoll und installiert Pakete, welche noch nicht heruntergeladen wurden, automatisch nach.

Der Download Link für MiKTeX ist in den Quellen vorhanden.\autocite[Vgl.][]{miktex} Hierbei kann der gewöhnliche \directAutocite{miktex}{Installer} verwendet werden. Anschließend führt das Installationsprogramm den Nutzer durch den gesamten Installationsprozess. Hierbei muss nichts besonderes beachtet werden.

\section{TexMaker}
TexMaker ist der Editor und damit die Schnittstelle zum Nutzer. Zwar ist es möglich, sämtliche Dokumente in Texteditoren wie Notepad++ zu erstellen und diese über die Eingabeaufforderung zu einem PDF (\nomenclature{PDF}{Portable Document Format}Portable Document Format) umzuwandeln, doch bietet TexMaker angenehme Funktionen für die erleichterte Nutzung.

Auch hier gibt es einen normalen Installer zum Download, welcher alle weiteren Schritte erklärt.\autocite[Vgl.][]{texmaker}

Zum Einrichten von TexMakter muss noch eine weitere Einstellung vorgenommen werden. Da \LaTeX\ zum Bilden von Referenzen, Generieren von Verzeichnissen etc zunächst temporäre Dateien erzeugt, welche erst beim nächsten compilen genutzt werden, müssen eine Reihe von Compile-Befehlen nacheinander aufgerufen werden. Hierzu kann in TexMaker unter Optionen die Funktion \enquote{TexMaker konfigurieren} aufgerufen werden. Unter \enquote{Schnelles Übersetzen} wird dann die Befehlsabfolge wie folgt eingetragen: \enquote{pdflatex -synctex=1 -interaction=nonstopmode \%.tex|"C:/Program Files/MikTex/miktex/bin/x64/makeindex.exe" \%.nlo -s nomencl.ist -o \%.nls|biber \%|pdflatex -synctex=1 -interaction=nonstopmode \%.tex|pdflatex -synctex=1 -interaction=nonstopmode \%.tex|"C:/Program Files/Adobe/Reader 11.0/Reader/AcroRd32.exe" \%.pdf}. Gegebenenfalles muss der Pfad zum Pdf-Viewer oder der makeindex.exe entsprechend der lokalen Installationsverzeichnisse werden.

Alternativ kann unter \enquote{TexMaker konfigurieren} das Verzeichnis von MakeIndex angepasst werden und unter Bib(la)TeX der Befehl \enquote{biber \%} eingegeben werden. Anschließend lässt sich unter \enquote{Schnelles Übersetzen} der obere Assistent aufrufen, bei dem man nun 1x pdfLatex, 1x MakeIndex, 1x BibTeX und 2x pdfLatex auswählt. 