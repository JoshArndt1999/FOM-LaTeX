\chapter{Gleichungen}
Ein großer Vorteil der Nutzung von \LaTeX\ ist die gute Darstellung von Formeln und Gleichungen. 
\section{Einzelne Gleichungen und Funktionen}
Formeln können entweder innerhalb einer Zeile angezeigt werden wie hier ($a+b^2=\frac{34}{\sqrt{4}}$), oder aber abgesetzt werden:
\[
    a+b^2=\frac{34}{\sqrt{4}}
\]
Außerdem können Formeln nummeriert werden.
\begin{equation}
    a+b^2=\frac{34}{\sqrt{4}}
\end{equation}
Der Code hierfür würde beispielsweise folgendermaßen lauten:
\begin{lstlisting}
%Formel in Zeile
$a+b^2=\frac{34}{\sqrt{4}}$

%abgesetzte Formel
\[
    a+b^2=\frac{34}{\sqrt{4}}
\]

%nummerierte Formel
\begin{equation}
    a+b^2=\frac{34}{\sqrt{4}}
\end{equation}
\end{lstlisting}

\section{Blöcke von Gleichungen und Funktionen}
Um ganze Blöcke von Gleichungen zu erstellen, in welchen die GLeichheitszeichen untereinander dargestellt werden, nutzt man folgende Funktion:

\begin{lstlisting}
\begin{eqnarray}
    y & = & d\\
    y & = & c_x+d\\
    y & = & b_x^{2}+c_x+d\\
    y & = & a_x^{3}+b_x^{2}
\end{eqnarray}
\end{lstlisting}
Es resultiert in folgendem Ergebnis:
\begin{eqnarray}
    y & = & d\\
    y & = & c_x+d\\
    y & = & b_x^{2}+c_x+d\\
    y & = & a_x^{3}+b_x^{2}
\end{eqnarray}

Alternativ können die Blöcke ohne Nummerierung und etwas anders ausgerichtet folgendermaßen erstellt werden:
\begin{lstlisting}
\[\begin{array}{lcr}
    y & \not= & d\\
    y_{a}& = & c_x+d\\
    y & = & b_x^{2}+c_x+d\\
    y & = & a_x^{3}+b_x^{2}
\end{array}\]
\end{lstlisting}
Dies sähe dann so aus:
\[\begin{array}{lcr}
    y & \not= & d\\
    y_{a}& = & c_x+d\\
    y & = & b_x^{2}+c_x+d\\
    y & = & a_x^{3}+b_x^{2}
\end{array}\]