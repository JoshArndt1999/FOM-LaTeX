% !TeX root = ../../main.tex

\chapter{Referenzen}\label{ch:referenzen}

Referenzen werden genutzt, wenn innerhalb des Textes auf Abbildungen, Tabellen oder Kapitel verwiesen werden soll. 
Um beim Einfügen eines weiteren Kapitels einen Verweis zu verfälschen (da Kapitel X nun etwas anderes ist), wird das Objekt über einen Marker (gesetzt durch \enquote{\textbackslash label\{marker\}}) referenziert. 
Hierzu gibt es vier wichtige Befehle:

\begin{description}
\item[{\textbackslash label\{marker\}}] erzeugt den Namen des Objektes, der einzigartig im Dokument sein muss.
\item[{\textbackslash ref\{marker\}}] erzeugt die Objektnummer, also ob es sich um Tabelle 1, 2 oder 3 etc. handelt.
\item[{\textbackslash pageref\{marker\}}] erzeugt die Seitennummer, auf welcher sich das Objekt befindet.
\item[{\textbackslash nameref\{marker\}}] stellt den Namen des Objektes dar, also z. B. die Bildüberschrift.
\end{description}

Ein Hinweis zu den Markern: Es empfiehlt sich, jeden Marker-Typ mit einem passenden Präfix zu versehen, wie z.\,B. \texttt{fig:} für Abbildungen, \texttt{tab:} für Tabellen, etc. Erstens erleichtert dies die Benennung, außerdem gibt verschiedene \LaTeX-Pakete, die aus dem entsprechenden Präfix Text erzeugen können (siehe \url{www.uweziegenhagen.de/?p=3711}).

\section{Beispiel}
Hier ist ein kleines Beispiel zur Nutzung von Referenzen.

\begin{lstlisting}
Die Abbildung \nameref{fig:fom_logo} hat die Nummer \ref{fig:fom_logo} und ist auf Seite \pageref{fig:fom_logo}
\end{lstlisting}
Ausgabe: \enquote{Die Abbildung \nameref{fig:fom_logo} hat die Nummer \ref{fig:fom_logo} und ist auf Seite \pageref{fig:fom_logo}}

\section{Warnung}

Dieses Kapitel ist lediglich dazu da, dich daran zu erinnnern, dass es niemals ein Kapitel mit nur einem Unterkapitel geben darf. ;-)