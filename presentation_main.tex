%!TEX TS-program = Arara
% arara: pdflatex: {shell: yes}

\documentclass[12pt,ngerman]{beamer}

\usepackage[utf8]{inputenc}
\usepackage[T1]{fontenc}
\usepackage{booktabs}
\usepackage{babel}
\usepackage{graphicx}
\usepackage{csquotes}
\usepackage{xcolor}

\author{Max Mustermann}
\title{FOM-Musterpräsentation}

\begin{document}

\begin{frame}

\maketitle

\end{frame}

\begin{frame}
\frametitle{Aufzählung}

\begin{itemize}
\item 
\item 
\item 
\item 
\item 
\item 
\end{itemize}

\end{frame}


\begin{frame}
\frametitle{Aufzählung}

\begin{enumerate}
\item 
\item 
\item 
\item 
\item 
\item 
\end{enumerate}

\end{frame}

\begin{frame}
\frametitle{Aufzählung zweispaltig}

\begin{columns}
\begin{column}{0.5\textwidth}
\begin{itemize}
\item 
\item 
\item 
\item 
\item 
\item 
\end{itemize}
\end{column}
\begin{column}{0.5\textwidth}
\begin{itemize}
\item 
\item 
\item 
\item 
\item 
\item 
\end{itemize}
\end{column}
\end{columns}

\end{frame}


\begin{frame}
\frametitle{Bild}
\framesubtitle{Unter-Überschrift}

\begin{center}
\begin{figure}
\rule{\textwidth}{0.5\textwidth}
\caption{Eine Abbildung}
\end{figure}
\end{center}


\end{frame}


\begin{frame}
\frametitle{Formeln}

Eine Formel \(a^2+b^2=c^2\) im laufenden Text.

Eine abgesetzte Formel 

\[a^2+b^2=c^2\]

Eine abgesetzte Formel mit Nummer

\begin{equation}
a^2+b^2=c^2
\end{equation}


\begin{equation}
x_{1, 2} = -\frac{p}{2} \pm \sqrt{ \left(\frac{p}{2}\right)^2 -q }
\end{equation}



\end{frame}




\end{document}