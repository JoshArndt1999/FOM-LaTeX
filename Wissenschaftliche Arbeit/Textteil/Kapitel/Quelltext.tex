% !TeX root = ../../main.tex

\chapter{Quelltext}

Wie ihr in der Anleitung nun schon einige Male gesehen habt, lässt sich auch Quelltext in die Arbeit integrieren. 
Um Quelltext innerhalb einer Zeile zu integrieren kann dieser folgendermaßen eingefügt werden: \befehl|\befehl?Beispielbefehl?|. Die Fragezeichen können hier mit jedem beliebigen Zeichen ersetzt werden, welches nicht in dem einzufügenden Befehl vorkommt.
Um ihn in einen eigenen Block zu schreiben kann alternativ der Quelltext in \befehl|\begin{lstlisting}| und \befehl|\end{lstlisting}| eingeramt werden.


Der Quelltext wird nun eingefügt und mit Zeilennummern versehen. 
Lasst euch nicht irritieren, wenn der Code vom Editor als Befehl registriert wird (insbesondere wenn es sich um einzufügenden \LaTeX -Code handelt.  
Im resultierenden PDF ist es trotzdem als eingefügter Quelltext vorhanden. 
Es ist also lediglich die Anzeige im Editor fehlerhaft.
