% !TeX root = ../../main.tex

\chapter{Auflistungen}

\section{Beschreibung}
Die Beschreibung (Description) ist zum Beispiel dazu da, einzelne Begriffe genauer zu definieren.
Ein Beispiel könnte so aussehen:
\begin{description}
    \item[Ein Stichpunkt] Hier wird der erste Stichpunkt beschrieben.
    \item[Noch ein Stichpunkt] und hier wird der zweite beschrieben. 
\end{description}

Im Quelltext wird es folgendermaßen umgesetzt:
\begin{lstlisting}
\begin{description}
    \item[Ein Stichpunkt] Hier wird der erste Stichpunkt beschrieben.
    \item[Noch ein Stichpunkt] und hier wird der zweite beschrieben. 
\end{description}
\end{lstlisting}

Soll die Beschreibung erst in der nächsten Zeile beginnen, so könnte man dies wie folgt erreichen:
\begin{description}
    \item[Ein Stichpunkt] \hfill \\
        Hier wird der erste Stichpunkt beschrieben.
    \item[Noch ein Stichpunkt]  \hfill \\
        und hier wird der zweite beschrieben. 
\end{description}

Im Quelltext wird es folgendermaßen umgesetzt:
\begin{lstlisting}
\begin{description}
    \item[Ein Stichpunkt] \hfill \\
        Hier wird der erste Stichpunkt beschrieben.
    \item[Noch ein Stichpunkt]  \hfill \\
        und hier wird der zweite beschrieben. 
\end{description}
\end{lstlisting}

\section{Itemize}
Itemize wird verwendet, um verschiedene unnummerierte Aufzählungen zu erzeugen. 
Um das Kapitel nicht mit zu viel Code zu füllen, werden wir hier auf längere Code-Beispiele verzichten. Grundsätzlich ist das Prinzip jedoch dasselbe wie bei \befehl|description| bereits gezeigt.

\subsection{Beispiel mit Standardauflistungszeichen}
\begin{itemize}
    \item erste Ebene
    \begin{itemize}
        \item zweite Ebene
        \begin{itemize}
            \item dritte Ebene
            \begin{itemize}
                \item vierte Ebene
            \end{itemize}
            \item wieder auf dritter Ebene 
            \item noch ein Eintrag 
        \end{itemize}
        \item hier ist die zweite Ebene
    \end{itemize}
    \item und hier die erste Ebene
\end{itemize}

\subsection{Beispiel mit bestimmten Auflistungszeichen}
\begin{itemize}
    \item[a)] Ein Stichpunkt
    \item[*)] Noch ein Stichpunkt
    \item[?)] Stichpunkt drei
\end{itemize}

\subsection{Dauerhafte Änderung der Auflistungszeichen}
Werden im Laufe der Arbeit dauerhaft andere Auflistungszeichen benötigt, so sollten diese nicht jedes mal neu als Parameter übergeben werden (Redundanz). 
In dem Fall kann man die Auflistungszeichen der verschiedenen Ebenen (hier die ersten 4) folgendermaßen anpassen:
\begin{lstlisting}
\renewcommand{\labelitemi}{$\bullet$}
\renewcommand{\labelitemii}{$\bullet$}
\renewcommand{\labelitemiii}{$\bullet$}
\renewcommand{\labelitemiv}{$\bullet$}
\end{lstlisting}
Statt \befehl|$\bullet$| kann hier das gewünschte Zeichen eingegeben werden. Hier wird lediglich der "normal" Auflistungs-Punkt genutzt.

\section{Aufzählung}
Bei einer Aufzählung werden die einzelnen Stichpunkte von eins bis n durchnummeriert.
\begin{enumerate}
    \item erste Ebene
    \begin{enumerate}
        \item zweite Ebene
        \begin{enumerate}
            \item dritte Ebene
            \begin{enumerate}
                \item vierte Ebene
            \end{enumerate}
        \end{enumerate}
    \end{enumerate}
\end{enumerate}
Sollen hier die Aufzählungen weiter angepasst werden und zum Beispiel die Zahlen immer arabisch dargestellt werden, so gibt es weitere Quellen, die hierrauf genauer eingehen.\indirekt{auflistungen}

\section{Kompaktere Auflistungen}

Die oben beschriebenen Auflistungen brauchen alle relativ viel vertikalen Platz -- Platz, den man nicht immer zur Verfügung hat. 
Mit dem paralist Paket gibt es ein Paket, das deutlich kompaktere Aufzählungen ermöglicht. 
Das Paket definiert die folgenden Auflistungen: \texttt{compactitem} als Ersatz für \texttt{itemize}, \texttt{compactenum} als Ersatz für \texttt{enumerate}, sowie \texttt{compactdesc} als Ersatz für \texttt{description}.

\begin{figure}[H] 
   \centering 
   \caption{Normale und kompakte \texttt{itemize} Umgebung} 
   \label{fig:compactItem} 
		\fbox{%
		\begin{minipage}[l]{0.48\textwidth}
		\begin{itemize}
			\item Ein
			\item Beispiel
			\item für eine
			\item normale
			\item \texttt{itemize}
			\item Umgebung
			\end{itemize}
		\end{minipage}}
		\fbox{%
		\begin{minipage}[l]{0.48\textwidth}
		\begin{compactitem}
			\item Ein
			\item Beispiel
			\item für eine
			\item kompakte
			\item \texttt{compactitem}
			\item Umgebung
		\end{compactitem}
		\end{minipage}}
   \par\bigskip  
   Quelle: eigene Darstellung 
\end{figure} 

\begin{figure}[H] 
   \centering 
   \caption{Normale und kompakte \texttt{enumerate} Umgebung} 
   \label{fig:compactItem} 
		\fbox{%
		\begin{minipage}[l]{0.48\textwidth}
		\begin{enumerate}
			\item Ein
			\item Beispiel
			\item für eine
			\item normale
			\item \texttt{enumerate}
			\item Umgebung
			\end{enumerate}
		\end{minipage}}
		\fbox{%
		\begin{minipage}[l]{0.48\textwidth}
		\begin{compactenum}
			\item Ein
			\item Beispiel
			\item für eine
			\item kompakte
			\item \texttt{compactenum}
			\item Umgebung
		\end{compactenum}
		\end{minipage}}
   \par\bigskip  
   Quelle: eigene Darstellung 
\end{figure} 

Wichtiger Hinweis: Mit der \texttt{Beamer}-Klasse ist das Paket leider nicht kompatibel.



