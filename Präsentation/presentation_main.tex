%!TEX TS-program = Arara
% arara: pdflatex: {shell: yes}
\documentclass[12pt,ngerman]{beamer}

\usepackage[utf8]{inputenc}
\usepackage[T1]{fontenc}
\usepackage{booktabs}
\usepackage{babel}
\usepackage{graphicx}
\usepackage{csquotes}
\usepackage{xcolor}
\usepackage{listings}

%%%%%%%%%%%%%%%%%%% Konfiguration
%%%% Farbdefinitionen
\definecolor{fomgruen}{RGB}{0,153,139}

% für Listings
\definecolor{hellgelb}{rgb}{1,1,0.8}
\definecolor{lightgelb}{rgb}{1,1,0.8}
\definecolor{colKeys}{rgb}{0,0,1}
\definecolor{colIdentifier}{rgb}{0,0,0}
\definecolor{colComments}{rgb}{1,0,0}
\definecolor{colString}{rgb}{0,0.5,0}

\lstset{%
    float=hbp,%
    basicstyle=\ttfamily\footnotesize, %
    identifierstyle=\color{colIdentifier}, %
    keywordstyle=\color{colKeys}, %
    stringstyle=\color{colString}, %
    commentstyle=\color{colComments}, %
    literate={fl}{{f{}l}}2,%
    columns=flexible, %
    tabsize=2, %
    frame=single, %
    upquote=true,%
    extendedchars=true, %
    showspaces=false, %
    showstringspaces=false, %
    numbers=left, %
    numberstyle=\tiny, %
    breaklines=true, %
    backgroundcolor=\color{hellgelb}, %
    breakautoindent=true, %
    captionpos=b%
}

% keine Symbole rechts unten
\setbeamertemplate{navigation symbols}{}

\setbeamercolor{title}{fg=fomgruen}
\setbeamercolor{frametitle}{fg=fomgruen}
\setbeamercolor{structure}{fg=fomgruen}

%%%%%%%%%%%%%%%%%%% Ende der Konfiguration

\author{Max Mustermann}
\title{FOM-Musterpräsentation}
\institute{FOM Hochschule für Oekonomie \& Management}

\begin{document}

\begin{frame}

\maketitle

\end{frame}

\begin{frame}
\frametitle{Aufzählung}

\begin{itemize}
\item 
\item 
\item 
\item 
\item 
\item 
\end{itemize}

\end{frame}


\begin{frame}
\frametitle{Aufzählung}

\begin{enumerate}
\item 
\item 
\item 
\item 
\item 
\item 
\end{enumerate}

\end{frame}

\begin{frame}
\frametitle{Aufzählung zweispaltig}

\begin{columns}
\begin{column}{0.5\textwidth}
\begin{itemize}
\item 
\item 
\item 
\item 
\item 
\item 
\end{itemize}
\end{column}
\begin{column}{0.5\textwidth}
\begin{itemize}
\item 
\item 
\item 
\item 
\item 
\item 
\end{itemize}
\end{column}
\end{columns}

\end{frame}


\begin{frame}
\frametitle{Bild}
\framesubtitle{Unter-Überschrift}

\begin{center}
\begin{figure}
\rule{\textwidth}{0.5\textwidth}
\caption{Eine Abbildung, Quelle: Maria Mustermann, 1995}
\end{figure}
\end{center}


\end{frame}


\begin{frame}
\frametitle{Formeln}

Eine Formel \(a^2+b^2=c^2\) im laufenden Text.

Eine abgesetzte Formel 

\[a^2+b^2=c^2\]

Eine abgesetzte Formel mit Nummer

\begin{equation}
a^2+b^2=c^2
\end{equation}


\begin{equation}
x_{1, 2} = -\frac{p}{2} \pm \sqrt{ \left(\frac{p}{2}\right)^2 -q }
\end{equation}

\end{frame}

\begin{frame}[containsverbatim] % wichtig!
\frametitle{Listings}

\begin{lstlisting}[language={Python},caption={Python Code},morekeywords={getLogger, propagate, setLevel}]
logger = logging.getLogger("Logfile.log")
logger.propagate = False
logger.setLevel(logging.DEBUG)
\end{lstlisting}
\end{frame}


\end{document}